The kinematic tree is an abstract representation of a collection of transforms. 
Unlike the rigid collection (section \ref{sec: rigid}), the kinematic tree is able to represent transforms in multiple datum frames. 
The tree can then use the network of frames to "lookup" a transform between any two connected nodes, as hinted at in figure \ref{fig: poses}.
This concept is widely used in robotics and therefore has a place in the prototyping package. 
There are several algorithms this class implements:
\begin{enumerate}
	\item Representation: Construct a tree representation given only the edges
	\item Lookup: apply breadth first graph search to find a path between any two nodes on the tree
	\item Root: use depth first search to express all frames in the base link frame or another specified frame on the tree
\end{enumerate}

\subsection{Representation}
Frames in a kinematic tree must have a child frame, a parent frame, or both. 
Frame $T$ is described as:
\begin{equation}
	T_i^{p(i)} = \begin{bmatrix}
		R_i^{p(i)} & t_i^{p(i)} \\
		\bf{0} & 1
	\end{bmatrix}
\end{equation}
where $i$ is the child frame, and $p(i)$ is the parent frame. 
The kinematic tree has a root, which we can also call the "base link". 
The challenge with the kinematic tree is that we are only given the transforms, which are the edges in the graph. 
So we need to create a tree representation from just the transforms. 
Transforms contain the parent and child of the edge, and are therefore directed. 
Thus we want a representation that is easily able to express the directed nature of the graph. 

The importance of the edges in the kinematic tree suggests that an incidence list is the best representation for this situation. 
For example, say we have a tree as shown in figure \ref{fig: tree}.

\begin{figure}
	\centering
	\includegraphics[width=0.5\textwidth]{images/example_tree.png}
	\caption{Example tree}
	\label{fig: tree}
\end{figure}

This takes the form:
\begin{align}
	base &: \big[ T^{base}_1, T^{base}_2\big] \\
	1 &: \big[ T^1_3, T^1_{base} \big]\\
	2 &: \big[ T^2_{base}\big]\\
	3 &: \big[ T_4^3, T_5^3 \big] \\
	4 &: \big[ T_3^4\big] \\
	5 &: \big[ T_3^5\big] \\
\end{align}
We can improve memory storage by using a dictionary (hash map) in order to only save the relevant edges. 
The construction algorithm is shown in algorithm \ref{alg: inc_list}.

\begin{algorithm}
	\DontPrintSemicolon
	\KwIn{Edge set $T$}
	\KwOut{Hash map where frames are the keys and the transforms they are incident with are values}
	tree $\gets \{\}$

	\tcp{hash.insert(key : value)}
	tree.insert(root : $\varnothing$)

	\For{$t \in T$}{$p \gets t.parent$

		$c \gets t.child$

		\tcp{process parents first}
		\If{$p \neq NULL$}{
			\eIf{$p \in$ tree}{
				\tcp{we store list of transforms touching $p$}
				tree[$p$].append($t$)
			}{
				tree.insert($p : \varnothing$)
			}
		}

		\tcp{process children}
		\If{$c \neq NULL$}{
			\eIf{$c \in $ tree}{
				\tcp{store whether the edge is forwards (used later)}
				($t^{-1}).forwards \gets false$

				tree[$c$].append($t^{-1}$)
			}{
				tree.insert($c : \varnothing$)
			}

		}
	}

	\Return{tree}
	\caption{Incidence list representation for the Kinematic Tree}
	\label{alg: inc_list}
\end{algorithm}

Once the incidence list is constructed, then we can use it to query the tree to find paths between frames.
The incidence list can keep track of the direction by assigning a flag to each edge specifying whether it is backwards or forwards. 
This will be useful for the rooting algorithm, shown later. 

\subsection{Lookup path between frames}

Now say we wanted to know the transform between two frames $i, j$ on the tree that aren't connected by an existing edge. 
We can find this edge, $T_i^j$ by finding the path between the two frames, and then multiplying the transforms we pass through. 

\begin{figure}
	\centering
	\includegraphics[width=\textwidth]{images/lookup_path.png}
	\caption{Lookup path between frames}
	\label{fig: lookup}
\end{figure}